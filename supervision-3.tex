\documentclass{supervision}
\usepackage{course}

\Supervision{3}
\begin{document}
  \begin{questions}
    \question
      \begin{parts}
        \part[4] Describe what calling conventions are and what they are used
          for. Illustrate your answer using an example in pseudo assembly code.
          \begin{solution}
            A calling convention is a scheme for how parameters are passes to
            a function from the caller and how they return a result.

            Different calling conventions may differ in:
            \begin{itemize}
              \item Where parameters, return values and return addresses are
                placed (registers, call stack etc.)
              \item Order in which arguments are passed
              \item How a return value is returned from the callee to the caller
              \item Whether the caller or callee is responsible for cleaning up
                and setting up a stack frame.
            \end{itemize}

            C uses the CDECL calling convention (right to left and the
            calling function cleans the stack).

            Consider the following function:

            \begin{code}{C}
              int example(int a, int b) {
                return a + b;
              }
            \end{code}

            called by:

            \begin{code}{C}
              x = example(2, 3)
            \end{code}

            These would produce the following assembly listings repsectively:

            \begin{code}{assembly}
              example:
                push ebp
                mov ebp, esp
                mov eax, [ebp + 8]
                mov edx, [ebp + 12]
                add eax, edx
                pop ebp
                ret
            \end{code}

            \begin{code}{assembly}
              push 3
              push 2
              call example
              add esp, 8
            \end{code}

            On entry to example, $esp$ points to the return address pushed onto
            the stack by the call instruction.
          \end{solution}

        \part[2] Explain the difference between calling conventions and
          evaluation strategies.
          \begin{solution}
            Evaluation strategy is seen as part of the language where as
            calling convention is seen as implementation details. The
            evaluation strategy may influence or even dictate some parts of the
            calling convention but they are considered separate.

            Evaluation strategy determines when and in what order to evaluate
            the arguments of a function call and what kind of value to pass.
          \end{solution}

        \part[2] A compiler author is writing a code generator which targets
          the C language. His intention is to implement CPS so that each
          function from his source language is translated into a C function
          which pops its continuation off an argument stack and then calls the
          continuation at the end. For “convenience”, he defines the following
          macro:

          \begin{code}{C++}
            #define JUMP(x) (*x)()
          \end{code}

          He then uses the macro to jump to continuations:

          \begin{code}{C++}
            void foo() {
                void* cont = *Stack--; // pop continuation off the stack
                // code for foo
                JUMP(&cont); // cont is the continuation
            }
          \end{code}

          Explain potential pitfalls with this implementation of CPS.

          \begin{solution}
            Each continutation is called from within the callee's scope and thus
            the stack will grow and grow as the program executes. Eventually
            stack overflow will occur.
          \end{solution}
        \part[2] The compiler author reconsiders his approach in light of your
          answer to the previous question and changes the signatures of all C
          functions to e.g.:

          \begin{code}{C++}
            void* foo();
          \end{code}

          He also changes his macro to:

          \begin{code}{C++}
            #define JUMP(x) return(x)
          \end{code}

          Finally, he implements the following:

          \begin{code}{C++}
            int main(int argc, char** argv) {
                void* cont = &foo;
                while(1) { cont = (*cont)(); }
                return 0;
            }
          \end{code}

          Explain why this approach is better.

          \begin{solution}
            The continuations are popped off the stack and executed by a while
            loop and therefore since a continuation returns before the next one
            is executed, its stack frame is removed.
          \end{solution}

        \part[2] If the compiler author were to choose an assembly language as
          the target language for his compiler, how should he implement CPS
          then?
          \begin{solution}
            Every subroutine returns the address of its continuation with a
            main loop that just calls the return address (as in the C example).
          \end{solution}

      \end{parts}

    \question Recall that configurations in the STG-machine are 6-tuples
      consisting of:

      \begin{enumerate}
        \item the code to be executed
        \item the argument stack ($as$), which contains values
        \item the return stack ($rs$), which contains continuations
        \item the update stack ($us$), which contains update frames
        \item the heap ($h$), which contains closures
        \item the global environment ($\sigma$), which gives the addresses of
          all closures defined at top level
      \end{enumerate}

      How do we map these to a C program? We begin by looking at the heap. For
      this purpose, we will distinguish between two kinds of objects on the
      heap:

      \begin{itemize}
        \item \emph{Head normal forms} or \emph{values} are heap objects which
          can not be evaluated any further, and
        \item \emph{suspensions} or \emph{thunks} are heap objects which have
          not yet been evaluated
      \end{itemize}

      Collectively, we will refer to values and thunks as \emph{closures}.
      Traditionally, a closure consists of a pointer to the code of a suspended
      computation (i.e. just a pointer to some code), and the values of all
      free variables. For example, think about the Further Java exercises in
      which you created anonymous methods and classes. Those resulted in the
      creation of closures which contained copies of the variables that were
      captured by the anonymous methods or classes.

      In the STG-machine we use this representation for both, values and
      thunks. This corresponds to the way we represent closures in the
      operational semantics. I.e. the $Eval e \rho$ code component consists of
      an expression $e$ (the code) and the local environment $\rho$ (the free
      variables).

      \begin{parts}
        \part[2] Show how you would represent STG closures in a low-level
          systems language, such as C, and explain your answer.

          \begin{solution}
            I would represent them as a pointer to a code section and a pointer
            to an array of free variables.

            Since free variables are themselves closures, this will be an array
            of pointers to closures.

            When the code section is entered it sets the code pointer to a
            blackhole pointer that if executed causes an error (since if this
            happens then the value must depend on itself which is not valid).

            When the code section has evaluated, it overwrites itself with the
            resulting data.
          \end{solution}

        \part[4] It is possible to generate static closures for all global
          bindings. Generate closures and function prototypes (stubs) in C for
          all global bindings in the following STG program.

          \begin{code}{{}}
            nil  = {} \n {}  -> Nil {}
            succ = {} \n {x} -> +# {x, 1#}
            list = {} \n {}  -> Cons {5#, nil}
            map  = {} \n {f,xs} ->
                      case xs {} of
                        Nil {}      -> Nil {}
                        Cons {y,ys} -> let  fy = {f,y}  \n {} -> f {y}
                                           mfy = {f,ys} \n {} -> map {f,ys}
                                       in Cons {fy, mfy}
            main = {} \n {} -> map {succ, list}
          \end{code}

          For example, the C function corresponding to \lstinline|nil| may look
          like this:

          \begin{code}{C}
            void* nil_code() {
              // do nothing yet
              JUMP(NULL); // to make the compiler happy for now
            }
          \end{code}

          \begin{solution}
            \begin{code}{C}
              void* succ_code() {
                // code
                JUMP(*stack--);
              }

              void* list_code() {
                // code
                JUMP(*stack--);
              }

              void* map_code() {
                // code
                JUMP(*stack--);
              }

              void* main_code() {
                //code
                JUMP(*stack--);
              }

              Closure succ_closure = {
                &succ_code,
                {}
              };

              Closure list_closure = {
                &list_code,
                {}
              };

              Closure map_closure = {
                &map_code,
                {}
              };

              Closure main_closure = {
                &main_code,
                {}
              };
            \end{code}
          \end{solution}

        \part[2] Explain why there is no need to implement the global
          environment $\sigma$ when mapping the STG machine to C.
          \begin{solution}
            When mapping to C you can simply use the C global environment and
            let C handle it.
          \end{solution}

        \part[1] Closures for local bindings must be allocated dynamically at
          runtime because we cannot predict what their free variables will be.
          For this purpose, we require a heap that we can allocate memory on.
          Do not worry about freeing memory. Show how you would implement this
          in C.
          \begin{solution}
            \emph{I'm not comfortable enough with C for a full working example.}

            Use malloc to allocate space on the heap. In the previous question
            I used a pointer to an array as I wasn't sure if you could do it
            otherwise since the struct needs to have a fixed size, I saw
            something about a flexible array member but didn't really get it.

            So you would create a new struct then use malloc to allocate space
            for the pointer array.
          \end{solution}

        \part[1] How would you implement a test for whether there is free space
          on the heap?
          \begin{solution}
            Malloc will return a \lstinline|NULL| if there is insufficient
            space.
          \end{solution}

        \part[3] Suppose that a local binding $fy$ has a corresponding C
          function named $fy_code$ which captures two free variables. Show how
          you would allocate space on the heap for a corresponding closure and
          how you would initialise it, assuming that pointers to the closures
          for the two free variables are available in local variables named $f$
          and $y$.
          \begin{solution}
            \begin{code}{C}
              Closure* closure;
              closure->code = &fy_code;

              // needs to be on heap - not sure

              Closure *closures[2];

              closures[0] = f;
              closures[1] = y;

              return closure;
            \end{code}
          \end{solution}

      \end{parts}

    \question
      \begin{parts}
        \part It is possible to implement all three stacks ($as$, $rs$, and
          $us$) using one concrete stack and, for simplicity, that’s what we
          will do for now. Traditionally, stacks grow from higher memory
          addresses to lower memory addresses. Why?
          \begin{solution}
            The stack and the heap need to be able to grow dynamically and
            therefore the most flexible layout is for one to grow down from a
            high address and the other to grow up from the bottom.

            As to why it is heap up, stack down I can't think of any solid
            reasons for this - it seems to just be convention.
          \end{solution}

        \part Show how you would implement a stack in C, including sample code
          to push and pop items.
          \begin{solution}
            \codefile[c]{code/stack.c}
          \end{solution}

        \part The process of evaluating (“entering”) a closure involves jumping
          to the memory address indicated by the code pointer of a closure and
          pushing any arguments onto the stack. How would you ensure that the
          code component of the closure can find its free variables? How would
          you address the argument?
          \begin{solution}
            When you enter a closure you first push a pointer to it into a
            register (lets call it the closure_register). Then the code
            component can just look at this register.
          \end{solution}

        \part Explain what we mean by “boxed” and “unboxed” values. Give an
          example of both in the STG language.
          \begin{solution}
            An unboxed value is a primitive where as a boxed value is a
            minimal wrapper around a primitive type.

            Closures are an example of boxed, I'm not sure about unboxed.
          \end{solution}

        \part % e
          \begin{solution}
            \begin{code}{C}
              void* zero() {
                int_res = 0;
                JUMP(*stack--);
              }

              void* not() {
                if(int_res == 0) {
                  int_res = 1;
                } else {
                  int_res = 0;
                }

                JUMP(*stack--);
              }

              void main() {
                PUSH(not);
                zero();
              }
            \end{code}
          \end{solution}
        \part % f
          \begin{solution}
            \emph{Not sure}
          \end{solution}
      \end{parts}
    \question
      \begin{parts}
        \part[4] STG (as described in the original paper) uses the Copying
          Collection technique for garbage collection. Briefly explain how this
          works in general.
          \begin{solution}
            With copy collection you have two heaps, when the garbage collector
            runs it copies traverses the heap from the roots and copies across
            everything that is accessible into the other heap.
          \end{solution}

        \part[2] Copying Collection requires two heaps. How would you implement
          them space-efficiently?
          \begin{solution}
            I would use a two way heap (one that goes bottom up and one that
            goes top down), that way when you copy the objects across you can
            fill in the gaps left.
          \end{solution}

        \part[2] Copying memory from one heap to another is a slow process.
          Explain one garbage collection technique which could reduce the
          amount of memory that has to be copied.
          \begin{solution}
            You can have different heaps with different GC policies. An object
            that has survived several GC runs could be moved to a heap with a
            less aggressive GC policy.
          \end{solution}
      \end{parts}

  \end{questions}
\end{document}
