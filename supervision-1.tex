\documentclass{supervision}
\usepackage{course}

\Supervision{1}

\begin{document}
  \begin{questions}
    \section*{Theoretical Exercise Set 1}
    \question Discuss how to eliminate recursion given a mutually
      tail-recursive functions. Transform the following by hand.

      \begin{code}{ml}
        let rec is_even n =
            if n = 0
            then true
            else is_odd(n - 1)

        and is_odd n =
            if n = 0
            then false
            else is_even(n - 1)
      \end{code}

      \begin{solution}
        \begin{code}{ml}
          let is_even n =
              let rn = ref n
              in let _ = while !n > 2
                do
                  rn := !rn - 2
                done
              in if !rn = 0
                 then true
                 else false

        \end{code}
      \end{solution}

    \question Again by hand, eliminate tail recursion from
      \lstinline|fold_left|. Does your source-to-source transformation change
      the type of the function? If so, can you rewrite your code so that the
      type does not change?

      \begin{code}{ml}
        (* fold_left : ('a -> 'b -> 'a) -> 'a -> 'b list -> 'a *)
        let rec fold_left f accu l =
          match l with
              [] -> accu
          | a::l -> fold_left f (f accu a) l

        (* sum up a list *)
        let sum1 = fold_left (+) 0 [1;2;3;4;5;6;7;8;9;10]
      \end{code}

    \question Apply (by hand) the CPS transformation to the gcd code.

      Explain your results.

      \begin{code}{ml}
        let rec gcd(m, n) =
            if m = n
            then m
            else if m < n
                then gcd(m, n - m)
                else gcd(m - n, n)

        let gcd_test_1 = List.map gcd [(24, 638); (17, 289); (31, 1889)]
      \end{code}
      \begin{solution}
        \begin{code}{ml}
          let rec gcd_cps(cnt, m, n) =
              if m = n
              then cnt m
              else if m < n
                  then gcd_cps(cnt, m, n - m)
                  else gcd_cps(cnt, m - n, n)
        \end{code}
      \end{solution}

    \question Environments are treated as functions in \lstinline|interp_0.ml|.

      Can you transform these definitions, starting with defunctionalisation,
      and arrive at a list-based implementation of environments?

      \begin{code}{ml}
        (* update : ('a -> 'b) * ('a * 'b) -> 'a -> 'b *)
        let update(env, (x, v)) = fun y -> if x = y then v else env y

        (* mupdate : ('a -> 'b) * ('a * 'b) list -> 'a -> 'b *)
        let rec mupdate(env, bl) =
            match bl with
            | [] -> env
            | (x, v) :: rest -> mupdate(update(env, (x, v)), rest)

        (* env_empty : string -> 'a *)
        let env_empty = fun y -> failwith (y ^ " is not defined!\n")

        (* env_init : (string * 'a) list -> string -> 'a *)
        let env_init bl = mupdate(env_empty, bl)
      \end{code}
      \begin{solution}
        \begin{code}{ml}
          type ('a, 'b) obear = EMPTY | INHABITED of 'a * 'b * ('a, 'b) obear

          let apply bear y =
              match bear with
              | EMPTY -> failwith (y ^ " is not defined!\n")
              | INHABITED(x,v,rest) -> if x = y then v else apply rest y

          let rec mupdate(env, bl) =
              match bl with
              | [] -> env
              | (x, v) :: rest -> mupdate(INHABITED(x,v,env), rest)

          let env_init bl = mupdate(EMPTY, bl)





          let update(env, new_item) = new_item :: env

          let rec mupdate(env, bl) =
              match bl with
              | [] -> env
              | new_item :: rest -> mupdate(update(env, new_item), rest)

          let rec env_lookup(env, y) =
              match env with
              | [] -> failwith (y ^ " is not defined!\n")
              | (x, v) :: rest -> if x = y then v else env_lookup rest y
        \end{code}
      \end{solution}

    \question Below is the code for (uncurried) map, with a test using fib.
      Can you apply the CPS transformation to map to produce
      \lstinline|map_cps|? Will this \lstinline|map_cps| still work with fib?
      If not, what to do?

      \begin{code}{ml}
        (* map : ('a -> 'b) * 'a list -> 'b list *)
        let rec map(f, l) =
            match l with
            | [] -> []
            | a :: rest -> (f a) :: (map(f, rest))

        (* fib : int -> int *)
        let rec fib m =
            if m = 0
            then 1
            else if m = 1
                then 1
                else fib(m - 1) + fib (m - 2)

        let map_test_1 = map(fib, [0; 1; 2; 3; 4; 5; 6; 7; 8; 9; 10])
      \end{code}
      \begin{solution}
        \begin{code}{ml}
          let map(cnt, f, l) =
               match l with
               | [] -> cnt []
               | x::xs -> f (fun v -> map((fun v2 -> cnt (v :: v2)), f, (xs))) x
        \end{code}
      \end{solution}

    \section*{Practical Exercises}

    There are many possible extensions or modifications possible for the
    Slang.1 that be fun programming (nested comments, better verbose output,
    etc). However, I'll list only those exercises that I feel reinforce the main
    points of the lectures.

    They are listed from easiest to hardest.

    \question In this simple language, do we really need the
      return types of functions? Could we always infer them?

    \question Add division to the arithmetic operations. How will
      you handle division by zero?

    \question We have blurred the distinction between the defining language
      (OCaml) and the defined language (Slang.1) in several ways.  For example,
      Slang.1 integers are the ints supplied by OCaml on your machine. Now
      suppose that Slang.1 only allows 16-bit integers, where overflow or
      underflow should raise a run-time error.  Can you modify the
      implementation to match this? Are there additional built-in functions that
      you might want to add to such a language?

    \question Add mutual recursion to Slang1.  Such as

      \begin{code}{ml}
        let g(x : int) : int = ... f(e) ...
        and f(z :int) : int  = .... g(e') ...
        in ... end
      \end{code}

      This requires careful treatment of environments!

    \question Consider adding references to Slang.1. We could do this in
      several ways, from simplest to most complex:
      \begin{itemize}
        \item Allow references to be used only locally.
        \item Allow refs to be passed to functions and then updated by the
          called function.
        \item Allow refs to be returned by functions
      \end{itemize}

      Can \lstinline|interp_0.ml| be extended to handle each of these?
  \end{questions}
\end{document}
